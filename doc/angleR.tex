% -*- TeX:Rnw -*-
% ----------------------------------------------------------------
% .R Sweave file  ************************************************
% ----------------------------------------------------------------
%%
% \VignetteIndexEntry{}
% \VignetteDepends{}
% \VignettePackage{}
\documentclass[a4paper,12pt]{article}
\usepackage[cp1250]{inputenc}
%\usepackage[slovene]{babel}
\newcommand{\SVNRevision}{$ $Rev: 13 $ $}
%\newcommand{\SVNDate}{$ $Date:: 2009-04-1#$ $}
\newcommand{\SVNId}{$ $Id: angleR.Rnw 13 2009-04-14 23:16:27Z ABlejec $ $}
\input{abpkg}
\input{abcmd}
\input{abpage}
\usepackage{pgf,pgfarrows,pgfnodes,pgfautomata,pgfheaps,pgfshade}
\usepackage{amsmath,amssymb}
\usepackage{colortbl}
\usepackage{Sweave}
\input{mysweave}
\input{./figs/bla-concordance}
%\SweaveOpts{echo=false}
\setkeys{Gin}{width=0.7\textwidth}
\usepackage{lmodern}
\input{abfont}
%\SweaveOpts{keep.source=true}
%\setkeys{Gin}{width=0.8\textwidth} % set graphicx parameter
% ----------------------------------------------------------------
\begin{document}
%% Sweave settings for includegraphics default plot size (Sweave default is 0.8)
%% notice this must be after begin{document}
%%% \setkeys{Gin}{width=0.9\textwidth}
% ----------------------------------------------------------------
\title{Angle arithmetic: \pkg{angleR}}
\author{A. Blejec}
%\address{}%
%\email{}%
%
%\thanks{}%
%\subjclass{}%
%\keywords{}%

%\date{}%
%\dedicatory{}%
%\commby{}%
\maketitle
% ----------------------------------------------------------------
\begin{abstract}
A package to perform angle manipulations in degrees, minutes and seconds. Formatting, conversions and arithmetic functions are included. Might be useful for specialists in geodesy and geometry.
\end{abstract}
% ----------------------------------------------------------------
\tableofcontents

%\input{koti}
% -*- TeX:Rnw -*-
% ----------------------------------------------------------------
% .R Sweave part file  *******************************************
%
\usepackage[cp1250]{inputenc}
\begin{Schunk}
\begin{Sinput}
> rm(list=ls(all=TRUE))
> newAngle <- function(...) NULL
> as.degs <- function(...) NULL
> as.mins <- function(...) NULL
> as.secs <- function(...) NULL
\end{Sinput}
\end{Schunk}


% ----------------------------------------------------------------
\section{New angle}


\begin{Schunk}
\begin{Sinput}
> normalize.angle <- function(x){
+ x <- unclass(x)
+ x <- x[1]+x[2]/60+x[3]/3600
+ st <- floor(x)
+ min <- floor((x-st)*60)
+ sek <- (x-st-min/60)*3600
+ return(c(st,min,sek))
+ }
> normalize.angle(c(10,10,10))
\end{Sinput}
\begin{Soutput}
[1] 10 10 10
\end{Soutput}
\begin{Sinput}
> normalize.angle(c(10,70,71))
\end{Sinput}
\begin{Soutput}
[1] 11 11 11
\end{Soutput}
\begin{Sinput}
> alfa <- c(10,-20,-30)
> normalize.angle(alfa)
\end{Sinput}
\begin{Soutput}
[1]  9 39 30
\end{Soutput}
\begin{Sinput}
> round(normalize.angle(c(0,20,30))-normalize.angle(c(0,-20,-30)))
\end{Sinput}
\begin{Soutput}
[1]   1 -19   0
\end{Soutput}
\end{Schunk}

\begin{Schunk}
\begin{Sinput}
> newAngle <- function(d=0,m=0,s=0){
+ z <- sign(d)*(abs(d)+m/60+s/3600)
+ z <- structure(z,class=c("angle"))
+ return(z)
+ }
> str(newAngle(10,20,30))
\end{Sinput}
\begin{Soutput}
Class 'angle'  num 10.3
\end{Soutput}
\begin{Sinput}
> unclass(newAngle(10,70,70))
\end{Sinput}
\begin{Soutput}
[1] 11.18611
\end{Soutput}
\begin{Sinput}
> newAngle(-10,20,30)
\end{Sinput}
\begin{Soutput}
[1] -10.34167
attr(,"class")
[1] "angle"
\end{Soutput}
\begin{Sinput}
> newAngle(0,0,0)
\end{Sinput}
\begin{Soutput}
[1] 0
attr(,"class")
[1] "angle"
\end{Soutput}
\end{Schunk}

\section{Conversions}

\begin{Schunk}
\begin{Sinput}
> as.degs <- function(x){
+ if(inherits(x,"numeric")) return(structure(x,class=c("angle"),units="degs"))
+ if(inherits(x,"angle")) return(structure(x,class="angle",units="degs"))
+ 
+ }
> alfa <- newAngle(30,30,10)
> alfa
\end{Sinput}
\begin{Soutput}
[1] 30.50278
attr(,"class")
[1] "angle"
\end{Soutput}
\begin{Sinput}
> as.degs(alfa)
\end{Sinput}
\begin{Soutput}
[1] 30.50278
attr(,"class")
[1] "angle"
attr(,"units")
[1] "degs"
\end{Soutput}
\begin{Sinput}
> as.degs(-alfa)
\end{Sinput}
\begin{Soutput}
[1] -30.50278
attr(,"class")
[1] "angle"
attr(,"units")
[1] "degs"
\end{Soutput}
\begin{Sinput}
> as.degs(30.5)
\end{Sinput}
\begin{Soutput}
[1] 30.5
attr(,"class")
[1] "angle"
attr(,"units")
[1] "degs"
\end{Soutput}
\begin{Sinput}
> as.degs(newAngle(0,0,0))
\end{Sinput}
\begin{Soutput}
[1] 0
attr(,"class")
[1] "angle"
attr(,"units")
[1] "degs"
\end{Soutput}
\end{Schunk}

\begin{Schunk}
\begin{Sinput}
> as.mins <- function(x){
+ if(inherits(x,"numeric")) return(structure(x/60,class=c("angle"),units="mins"))
+ if(inherits(x,"angle")) return(structure(x,class="angle",units="mins"))
+ }
> alfa <- newAngle(30,30,10)
> alfa
\end{Sinput}
\begin{Soutput}
[1] 30.50278
attr(,"class")
[1] "angle"
\end{Soutput}
\begin{Sinput}
> as.mins(alfa)
\end{Sinput}
\begin{Soutput}
[1] 30.50278
attr(,"class")
[1] "angle"
attr(,"units")
[1] "mins"
\end{Soutput}
\end{Schunk}

\begin{Schunk}
\begin{Sinput}
> as.secs <- function(x){
+ if(inherits(x,"numeric")) return(structure(x/3600,class="angle",units="secs"))
+ if(inherits(x,"angle"))return(structure(x,class="angle",units="secs"))
+ }
> alfa <- newAngle(30,30,10.1)
> alfa
\end{Sinput}
\begin{Soutput}
[1] 30.50281
attr(,"class")
[1] "angle"
\end{Soutput}
\begin{Sinput}
> as.secs(alfa)
\end{Sinput}
\begin{Soutput}
[1] 30.50281
attr(,"class")
[1] "angle"
attr(,"units")
[1] "secs"
\end{Soutput}
\begin{Sinput}
> beta <- as.secs(1)
> beta
\end{Sinput}
\begin{Soutput}
[1] 0.0002777778
attr(,"class")
[1] "angle"
attr(,"units")
[1] "secs"
\end{Soutput}
\begin{Sinput}
> unclass(beta)
\end{Sinput}
\begin{Soutput}
[1] 0.0002777778
attr(,"units")
[1] "secs"
\end{Soutput}
\end{Schunk}



\begin{Schunk}
\begin{Sinput}
> as.angle <- function(x){
+ if(inherits(x,"numeric")) class(x) <- "angle"
+ return(x)
+ }
> as.angle(10.5)
\end{Sinput}
\begin{Soutput}
[1] 10.5
attr(,"class")
[1] "angle"
\end{Soutput}
\end{Schunk}




\begin{Schunk}
\begin{Sinput}
> is.degs <- function(x) attr(x,"units")=="degs"
> is.mins <- function(x) attr(x,"units")=="mins"
> is.secs <- function(x) attr(x,"units")=="secs"
> 
\end{Sinput}
\end{Schunk}


\begin{Schunk}
\begin{Sinput}
> c.angle <- function (..., recursive = FALSE)
+ structure(c(unlist(lapply(list(...), unclass))), class = "angle")
> #
> alfa
\end{Sinput}
\begin{Soutput}
[1] 30.50281
attr(,"class")
[1] "angle"
\end{Soutput}
\begin{Sinput}
> beta
\end{Sinput}
\begin{Soutput}
[1] 0.0002777778
attr(,"class")
[1] "angle"
attr(,"units")
[1] "secs"
\end{Soutput}
\begin{Sinput}
> str(c(alfa,beta))
\end{Sinput}
\begin{Soutput}
Class 'angle'  num [1:2] 3.05e+01 2.78e-04
\end{Soutput}
\end{Schunk}


\begin{Schunk}
\begin{Sinput}
> strsign.angle <- function(...) sign(unclass(...))
> #
> sign(c(as.angle(-10.5),as.angle(10)))
\end{Sinput}
\begin{Soutput}
[1] -1  1
attr(,"class")
[1] "angle"
\end{Soutput}
\end{Schunk}





\section{Format and print}

\subsection{\fct{zfill}}
Formatiranje celih �tevil na nekaj mest z vodilnimi ni�lami

\begin{Schunk}
\begin{Sinput}
> zfill <- function(x,digits=3){
+ #nd <- 0
+ #if(x>1) nd <- floor(log10(x))
+ #z <- paste(c(rep("0",digits-nd-1),x),collapse="")
+ z <- sprintf(paste("%0",digits,"d",sep=""),x)
+ return(z)
+ }
> sprintf("%04d",15)
\end{Sinput}
\begin{Soutput}
[1] "0015"
\end{Soutput}
\begin{Sinput}
> #
> zfill(9)
\end{Sinput}
\begin{Soutput}
[1] "009"
\end{Soutput}
\begin{Sinput}
> zfill(0)
\end{Sinput}
\begin{Soutput}
[1] "000"
\end{Soutput}
\begin{Sinput}
> zfill(30,2)
\end{Sinput}
\begin{Soutput}
[1] "30"
\end{Soutput}
\end{Schunk}

\clearpage
\begin{Schunk}
\begin{Sinput}
> format.angle <- function(x,units=NULL,m.small=1,s.dec=0,sep="",collapse=NULL,dd="� ",mm="' ",ss="''"){
+ if(!is.null(units)) attr(x,"units") <- units
+ if(is.null(attr(x,"units"))) {
+ sign <- c("-"," ","+")[2+sign(x)]
+ x <- abs(x)
+ d <- floor(x)
+ m <- floor((x-d)*60)
+ s <- (x-d-m/60)*3600
+ return(paste(sign,zfill(d,3),dd,zfill(m,2),mm,
+ round(s,m.small),ss,sep=sep,collapse=collapse))
+ }
+ if(attr(x,"units")=="dms") {
+ sign <- c("-"," ","+")[2+sign(x)]
+ x <- abs(x)
+ d <- floor(x)
+ m <- floor((x-d)*60)
+ s <- round((x-d-m/60)*36000)
+ return(paste(sign,zfill(d,3),zfill(m,2),
+ zfill(s,3),sep=sep,collapse=collapse))
+ }
+ if(attr(x,"units")=="degs") return(
+ paste(round(x,m.small),"�",sep=sep,collapse=collapse))
+ if(attr(x,"units")=="mins") return(
+ paste(round(x*60,m.small),"'",sep=sep,collapse=collapse))
+ if(attr(x,"units")=="secs") return(
+ paste(round(x*3600,m.small),"''",sep=sep,collapse=collapse))
+ }
> alfa <- newAngle(30,20,10.1)
> beta <- newAngle(50,40,1.1)
> format(alfa)
\end{Sinput}
\begin{Soutput}
[1] "+030� 20' 10.1''"
\end{Soutput}
\begin{Sinput}
> format(structure(c(alfa,-beta),class="angle",units="dms"))
\end{Sinput}
\begin{Soutput}
[1] "+03020101" "-05040011"
\end{Soutput}
\begin{Sinput}
> format(alfa,"mins")
\end{Sinput}
\begin{Soutput}
[1] "1820.2'"
\end{Soutput}
\begin{Sinput}
> format(as.degs(alfa))
\end{Sinput}
\begin{Soutput}
[1] "30.3�"
\end{Soutput}
\begin{Sinput}
> format(as.mins(alfa))
\end{Sinput}
\begin{Soutput}
[1] "1820.2'"
\end{Soutput}
\begin{Sinput}
> format(as.secs(alfa))
\end{Sinput}
\begin{Soutput}
[1] "109210.1''"
\end{Soutput}
\begin{Sinput}
> format(alfa,"dms")
\end{Sinput}
\begin{Soutput}
[1] "+03020101"
\end{Soutput}
\end{Schunk}

\begin{Schunk}
\begin{Sinput}
> print.angle <- function(x){
+ if(inherits(x,"angle")) print(noquote(format(x)))
+ }
> alfa <- newAngle(30,20,10)
> print(alfa)
\end{Sinput}
\begin{Soutput}
[1] +030� 20' 10''
\end{Soutput}
\begin{Sinput}
> alfa
\end{Sinput}
\begin{Soutput}
[1] +030� 20' 10''
\end{Soutput}
\begin{Sinput}
> as.degs(alfa)
\end{Sinput}
\begin{Soutput}
[1] 30.3�
\end{Soutput}
\begin{Sinput}
> as.mins(alfa)
\end{Sinput}
\begin{Soutput}
[1] 1820.2'
\end{Soutput}
\begin{Sinput}
> as.secs(alfa)
\end{Sinput}
\begin{Soutput}
[1] 109210''
\end{Soutput}
\begin{Sinput}
> beta <- newAngle(50,40,1.1)
> print(c(alfa,-beta))
\end{Sinput}
\begin{Soutput}
[1] +030� 20' 10''  -050� 40' 1.1''
\end{Soutput}
\end{Schunk}





\clearpage
\section{Arithmetic}

Angles are internally saved as decimal degrees, so usual arithmetic functions apply!

\subsection{\fct{*.angle}}
\begin{Schunk}
\begin{Sinput}
> (alfa <- newAngle(30,10,20))
\end{Sinput}
\begin{Soutput}
[1] +030� 10' 20''
\end{Soutput}
\begin{Sinput}
> (beta <- newAngle(20,20,50))
\end{Sinput}
\begin{Soutput}
[1] +020� 20' 50''
\end{Soutput}
\begin{Sinput}
> beta*2
\end{Sinput}
\begin{Soutput}
[1] +040� 41' 40''
\end{Soutput}
\begin{Sinput}
> beta+beta
\end{Sinput}
\begin{Soutput}
[1] +040� 41' 40''
\end{Soutput}
\begin{Sinput}
> 2*beta
\end{Sinput}
\begin{Soutput}
[1] +040� 41' 40''
\end{Soutput}
\begin{Sinput}
> beta*0.5
\end{Sinput}
\begin{Soutput}
[1] +010� 10' 25''
\end{Soutput}
\end{Schunk}

\clearpage
\subsection{\fct{+.angle}}
\begin{Schunk}
\begin{Sinput}
> (alfa <- newAngle(30,10,20))
\end{Sinput}
\begin{Soutput}
[1] +030� 10' 20''
\end{Soutput}
\begin{Sinput}
> (beta <- newAngle(20,20,50))
\end{Sinput}
\begin{Soutput}
[1] +020� 20' 50''
\end{Soutput}
\begin{Sinput}
> alfa+beta
\end{Sinput}
\begin{Soutput}
[1] +050� 31' 10''
\end{Soutput}
\begin{Sinput}
> as.degs(alfa)
\end{Sinput}
\begin{Soutput}
[1] 30.2�
\end{Soutput}
\end{Schunk}
\clearpage
\subsection{\fct{-.angle}}
\begin{Schunk}
\begin{Sinput}
> (alfa <- newAngle(30,10,20))
\end{Sinput}
\begin{Soutput}
[1] +030� 10' 20''
\end{Soutput}
\begin{Sinput}
> (beta <- newAngle(20,20,50))
\end{Sinput}
\begin{Soutput}
[1] +020� 20' 50''
\end{Soutput}
\begin{Sinput}
> -beta
\end{Sinput}
\begin{Soutput}
[1] -020� 20' 50''
\end{Soutput}
\begin{Sinput}
> alfa-beta
\end{Sinput}
\begin{Soutput}
[1] +009� 49' 30''
\end{Soutput}
\begin{Sinput}
> alfa-alfa
\end{Sinput}
\begin{Soutput}
[1]  000� 00' 0''
\end{Soutput}
\begin{Sinput}
> beta+beta
\end{Sinput}
\begin{Soutput}
[1] +040� 41' 40''
\end{Soutput}
\begin{Sinput}
> 2*beta
\end{Sinput}
\begin{Soutput}
[1] +040� 41' 40''
\end{Soutput}
\end{Schunk}

\clearpage
\subsection{\fct{/.angle}}
\begin{Schunk}
\begin{Sinput}
> (alfa <- newAngle(30,15,25))
\end{Sinput}
\begin{Soutput}
[1] +030� 15' 25''
\end{Soutput}
\begin{Sinput}
> (beta <- newAngle(20,25,55))
\end{Sinput}
\begin{Soutput}
[1] +020� 25' 55''
\end{Soutput}
\begin{Sinput}
> beta/2
\end{Sinput}
\begin{Soutput}
[1] +010� 12' 57.5''
\end{Soutput}
\begin{Sinput}
> beta/beta
\end{Sinput}
\begin{Soutput}
[1] +001� 00' 0''
\end{Soutput}
\begin{Sinput}
> alfa/beta
\end{Sinput}
\begin{Soutput}
[1] +001� 28' 51.1''
\end{Soutput}
\begin{Sinput}
> beta/0.5
\end{Sinput}
\begin{Soutput}
[1] +040� 51' 50''
\end{Soutput}
\begin{Sinput}
> newAngle(180,0,0)/newAngle(45,0,0)
\end{Sinput}
\begin{Soutput}
[1] +004� 00' 0''
\end{Soutput}
\begin{Sinput}
> newAngle(180,0,0)-3*newAngle(45,0,0)
\end{Sinput}
\begin{Soutput}
[1] +045� 00' 0''
\end{Soutput}
\begin{Sinput}
> newAngle(180,0,0)-2*newAngle(45,0,0)
\end{Sinput}
\begin{Soutput}
[1] +090� 00' 0''
\end{Soutput}
\begin{Sinput}
> newAngle(180,0,0)-2*newAngle(445,0,0)
\end{Sinput}
\begin{Soutput}
[1] -710� 00' 0''
\end{Soutput}
\begin{Sinput}
> newAngle(180,0,0)+2*newAngle(445,0,0)
\end{Sinput}
\begin{Soutput}
[1] +1070� 00' 0''
\end{Soutput}
\end{Schunk}

\clearpage
\section{Calculations}

\begin{Schunk}
\begin{Sinput}
> alfa
\end{Sinput}
\begin{Soutput}
[1] +030� 15' 25''
\end{Soutput}
\begin{Sinput}
> beta
\end{Sinput}
\begin{Soutput}
[1] +020� 25' 55''
\end{Soutput}
\begin{Sinput}
> alfa-beta+2*(beta/2)-alfa
\end{Sinput}
\begin{Soutput}
[1]  000� 00' 0''
\end{Soutput}
\begin{Sinput}
> beta-2*(beta/2)
\end{Sinput}
\begin{Soutput}
[1]  000� 00' 0''
\end{Soutput}
\begin{Sinput}
> alfa-alfa
\end{Sinput}
\begin{Soutput}
[1]  000� 00' 0''
\end{Soutput}
\begin{Sinput}
> -alfa
\end{Sinput}
\begin{Soutput}
[1] -030� 15' 25''
\end{Soutput}
\begin{Sinput}
> -alfa+newAngle(0,0,0)
\end{Sinput}
\begin{Soutput}
[1] -030� 15' 25''
\end{Soutput}
\begin{Sinput}
> -alfa+alfa
\end{Sinput}
\begin{Soutput}
[1]  000� 00' 0''
\end{Soutput}
\begin{Sinput}
> -alfa+beta-2*(beta/2)
\end{Sinput}
\begin{Soutput}
[1] -030� 15' 25''
\end{Soutput}
\begin{Sinput}
> -alfa+beta-2*(beta/2)+alfa
\end{Sinput}
\begin{Soutput}
[1]  000� 00' 0''
\end{Soutput}
\begin{Sinput}
> alfa+beta
\end{Sinput}
\begin{Soutput}
[1] +050� 41' 20''
\end{Soutput}
\begin{Sinput}
> alfa-beta
\end{Sinput}
\begin{Soutput}
[1] +009� 49' 30''
\end{Soutput}
\begin{Sinput}
> beta+alfa
\end{Sinput}
\begin{Soutput}
[1] +050� 41' 20''
\end{Soutput}
\begin{Sinput}
> 7*alfa
\end{Sinput}
\begin{Soutput}
[1] +211� 47' 55''
\end{Soutput}
\end{Schunk}




% ----------------------------------------------------------------
%\bibliographystyle{chicago}
%\addcontentsline{toc}{section}{\refname}
%\bibliography{ab-general}
%--------------------------------------------------------------

%\clearpage
%\appendix
%\phantomsection\addcontentsline{toc}{section}{\appendixname}
%\section{\R\ funkcije}
%\input{}

\clearpage
\section*{SessionInfo}
Windows 7 x64 (build 7601) Service Pack 1 \begin{itemize}\raggedright
  \item R version 2.14.1 (2011-12-22), \verb|x86_64-pc-mingw32|
  \item Locale: \verb|LC_COLLATE=Slovenian_Slovenia.1250|, \verb|LC_CTYPE=Slovenian_Slovenia.1250|, \verb|LC_MONETARY=Slovenian_Slovenia.1250|, \verb|LC_NUMERIC=C|, \verb|LC_TIME=Slovenian_Slovenia.1250|
  \item Base packages: base, datasets, graphics, grDevices,
    methods, splines, stats, utils
  \item Other packages: Hmisc~3.9-2, patchDVI~1.8.1584,
    survival~2.36-10
  \item Loaded via a namespace (and not attached):
    cluster~1.14.1, grid~2.14.1, lattice~0.20-0, tools~2.14.1
\end{itemize}

%\subsection*{View as vignette}
%Project files can be viewed by this code:
%
%<<vignette,eval=false>>=
%projectName <- ""
%mainFile <- ""
%library(tkWidgets)
% getRootPath <- function() {
% fp <- (strsplit(getwd(), "/"))[[1]]
% file <- file.path(paste(fp[-length(fp)], collapse = "/"))
% return(file)
% }
% fileName <- function(name="bla",ext="PDF") paste(name,ext,sep=".")
%
% openPDF(file.path(getRootPath(),"doc",fileName(mainFile,"PDF")))
% viewVignette("viewVignette", projectName, file.path("../doc",fileName(mainFile,"RNW")))
%@


\vfill
\hrule
\vspace{2pt}
\footnotesize{
Revision \SVNId\hfill (c) A. Blejec%\input{../_COPYRIGHT.}
%\SVNRevision ~/~ \SVNDate
}



\end{document}
% ----------------------------------------------------------------

